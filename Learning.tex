\documentclass[9pt]{article}

\usepackage{notes}

\setcitestyle{authoryear}

\begin{document}
\title{Notes on General Learning}
\author{Ermo Wei}
\date{}

\maketitle

\tableofcontents
\hypersetup{colorlinks=blue}

\clearpage 

\ItemTitle{activation}{Activation Function} Activation Function are important for neural network to have non-linearity. The commonly used activation function are summarize below. Most of them are typically combined with an affine transformation $\matr{a} = \matr{b} + \matr{W}\matr{x}$ and applied element-wise.
\[
	\matr{h} = f(\matr{a}) \rightarrow h_i = f(a_i) = f(b_i + \matr{W_{i,:}} *\matr{x})
\]
\begin{itemize}
\item Sigmoid\\
\[
f(a) = \frac{1}{1+e^{-a}}
\]
\item Hyperbolic tangent
\[
f(a) = \tanh(a)
\]
\item Softmax
\item Rectifier or rectified linear unit (ReLU)
\[
f(a) = \max(0,a)
\]
\item Maxout
\item Softplus\\
A smooth version of the rectifier, basically the same shape. But this function has differentiability and non-zero derivative everywhere.
\[
f(a) = \log(1+e^a)
\]
\end{itemize}

\ItemTitle{bp}{Back Propagation} Let's start with linear neurons (also called linear filters). Here we first make two assumptions:
\begin{itemize}
\item The neuron has a real- valued output which is a weighted sum of its inputs
\item The aim of learning is to minimize the square error summed over all training cases.
\end{itemize}  
The model of the linear neuron is like this :
\[
y = \sum w_i x_i = \matr{w}^T\matr{x}
\]
Since we are dealing with a simple linear neuron, we first derive the special case of back propagation algorithm, called \textit{delta rule}, which is a gradient descent learning rule for updating the weights of single layer neural network.
\begin{enumerate}
\item We first define the error of our training cases:
\[
	E = \frac{1}{2} \sum_{n \in \text{training set}} (t^n-y^n)^2
\]
\item Now we take the derivative of the error with respect to the weights
\[
\begin{split}
	\frac{\partial E}{\partial w_i} = &\ \frac{1}{2}\sum_n \frac{dE^n}{dy^n}\frac{\partial y^n}{\partial w_i} \\
	= & - \sum_n x_i^n (t^n-y^n)
\end{split}
\]
\item The \textbf{batch} delta rule changes the weights in proportion to their error derivatives summed over all training cases
\[
	\Delta w_i = - \epsilon \frac{\partial E}{\partial w_i} = \sum_n \epsilon x_i^n (t^n-y^n)
\]
\end{enumerate}

One thing to be notice here, we can get as close as we desire to the best answer by making the learning rate small enough

Ok, since we have the simplest neuron, let's making things little interesting by adding the logistic function. And now, our model become this:
\[
\begin{split}
z = &\ b + \sum_i w_i x_i\\
y = &\ \frac{1}{1+e^{-z}}
\end{split}
\]
their corresponding derivatives are 
\[
\begin{split}
\frac{\partial z}{\partial w_i} = \ x_i  \,\text{,}&\, \frac{\partial z}{\partial x_i} = \ w_i \\
\frac{dy}{dz} = &\ y(1-y)
\end{split}
\]
Thus, take the derivative of $y$ with respect to $w_i$ is
\[
\frac{\partial y}{\partial w_i} = \frac{dy}{dz} \frac{\partial z}{\partial w_i} = x_i y (1-y)
\]
and 
\[
\frac{\partial E}{\partial w_i} = \sum \frac{\partial E}{\partial y^n} \frac{\partial y^n}{\partial w_i} = -\sum_n x_i^n y^n(1-y^n)(t^n-y^n)
\]
In this equation, the first and last terms come from the delta rule, and the term in the middle is the slope of logistic

\Working

\ItemTitle{lstm}{Long-Short Term Memory} \Working

\ItemTitle{universal}{Universal Approximator} 
Accroding to \citep{hornik1989multilayer}, Multilayer feedforward networks are universal approximators. Single hidden layer  feedforward networks can approximate any measurable function arbitrarily well regardless of the activation function $\Psi$, the dimension of the input space $r$, and the input space environment $\mu$
\begin{enumerate}
\item universal approximators: standard multilayer feedforward networks are capable of approximating
any measurable function to any desired degree of accuracy
\item there are no theoretical constraints for the success of
feedforward networks
\item lack of success is due to inadequate learning, insufficient number of hidden units or the lack of a deterministic relationship between input and target
\item rate of convergence as the number of hidden units grows
\item rate of increase of the number of hidden units as the input dimension increases for a fixed accuracy
\end{enumerate}
\bibliography{Learning}
\end{document}




